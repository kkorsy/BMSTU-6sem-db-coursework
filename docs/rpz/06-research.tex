\chapter{Исследовательский раздел}

\section{Описание исследования}

Целью является исследование влияния наличия индекса на время выполнения запроса к базе данных.

Для проведения исследования была выбрана таблица \textit{Users}, для индексации --- поле \textit{u\_name}.
Время выполнения запроса измерялось с помощью втроенной в PostgreSQL команды EXPLAIN~\cite{explain}.

В листинге~\ref{lst:explain} представлен скрипт для анализа времени выполнения запроса.
\begin{lstlisting}[label={lst:explain}, caption={Скрипт для анализа времени выполнения запроса}]
EXPLAIN ANALYSE
SELECT * FROM users
ORDER BY u_name;
\end{lstlisting}

В листинге~\ref{lst:index} приведен скрипт для создания индекса.
\begin{lstlisting}[label={lst:index}, caption={Скрипт для создания индекса}]
CREATE INDEX users_name_index
ON users USING btree(u_name);
\end{lstlisting}

Технические характеристики устройства, на котором выполнялось исследование представлены далее:
\begin{itemize}[label={---}]
	\item операционная система: Windows 11, x64;
	\item оперативная память: 8 Гб;
	\item процессор: AMD Ryzen 5 5500U с видеокартой Radeon Graphics 2.10~ГГц.
\end{itemize}

\section{Результаты исследования}

В таблице~\ref{tbl:time_mes} представлены результаты замеров времени в миллисекундах.
Каждый результат замеров является усредненным значением времени выполнения запроса к базе данных.

\begin{table}[H]
    \begin{center}
        \begin{threeparttable}
            \captionsetup{justification=raggedright, singlelinecheck=off}
            \caption{Результаты замеров времени}
            \label{tbl:time_mes}
            \begin{tabular}{|r|r|r|}
                \hline
                Количество записей & Без индексации, мс & С индексацией, мс \\ \hline
                1 & 0.0265 & 0.0170 \\ \hline
                5001 & 17.1516 & 2.2535 \\ \hline
                10001 & 33.4603 & 4.2925 \\ \hline
                15001 & 48.8944 & 6.5388 \\ \hline
                20001 & 66.3264 & 8.9383 \\ \hline
                25001 & 79.9716 & 11.1433 \\ \hline
                30001 & 101.5194 & 15.2185 \\ \hline
                35001 & 143.2783 & 17.4035 \\ \hline
                40001 & 159.8446 & 18.7491 \\ \hline
                45001 & 171.9279 & 20.6471 \\ \hline
                50001 & 191.5169 & 25.4068 \\ \hline
                55001 & 211.8095 & 25.9988 \\ \hline
                60001 & 227.1767 & 28.1881 \\ \hline
                65001 & 247.2775 & 33.2508 \\ \hline
                70001 & 301.8896 & 34.5799 \\ \hline
                75001 & 285.8743 & 37.7099 \\ \hline
                80001 & 304.3699 & 39.7253 \\ \hline
                85001 & 327.2190 & 43.0365 \\ \hline
                90001 & 341.1442 & 47.4883 \\ \hline
                95001 & 369.6379 & 48.2742 \\ \hline
                100001 & 413.4842 & 53.0227 \\ \hline
            \end{tabular}
        \end{threeparttable}
    \end{center}
\end{table}

Визуализация результатов замеров представлена на рисунке~\ref{img:graph-research}.

\includeimage
    {graph-research}
    {f}
    {H}
    {0.9\textwidth}
    {Зависимость времени выполнения запроса к базе данных от количества записей}

Была проведена интерполяция полиномом первой степени. Полученные интерполянты:
$$ y_1 = 0.0040x - 8.25, y_2 = 0.0005x - 1.28$$

\section*{Вывод}

В результате исследования было получено, что наличие индекса уменьшает время выполнения запроса к базе данных.

