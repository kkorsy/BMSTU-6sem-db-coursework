\chapter{Аналитический раздел}

В данном разделе будет проведен анализ предметной области, анализ известных решений, будет формализована поставленная задача, а также данные, подлежащие хранению в проектируемой базе данных.


\section{Анализ предметной области}

Сериал --- многосерийный фильм с несколькими сюжетными линиями.
Существует широкий спектр жанров сериалов, включая драму, комедию, фантастику, ужасы, триллеры и многое другое. 
Каждый жанр имеет свои особенности и привлекает определенную аудиторию.

Некоторые сериалы становятся популярными и получают высокие оценки от зрителей и критиков. 
Рейтинг сериала может быть определен на основе оценок, просмотров, наград и обсуждений в СМИ.
Помимо оценок, зрители могут оставлять отзывы, доступные другим зрителям, о просмотренных сериалах.

Сериалы могут иметь различное количество серий (эпизодов) и сезонов. 
Продолжительность отдельной серии и их количество в сезоне также различно для каждого сериала.

В сериалах задействованы актеры, которые играют роли различных персонажей. 
Некоторые сериалы имеют известных актеров, которые привлекают зрителей своим именем и талантом.

Сериалы требуют профессиональной работы съемочной группы, включая режиссера, оператора, художника по костюмам, декоратора и других специалистов.
Технические аспекты, такие как качество съемки, монтаж, спецэффекты и звук, играют важную роль в создании атмосферы сериала.

\section{Анализ известных решений}

Для сравнения были выбраны следующие критерии:
\begin{enumerate}
    \item поиск сериала с указанными параметрами;
    \item наличие истории просмотров;
    \item добавление сериала в избранное;
    \item возможность сравнения выбранных сериалов. Под сравнением понимается просмотр таблицы, в которой столбцами являются названия выбранных сериалов, а строками --- их характеристики, такие как жанр, дата выхода, рейтинг и т. д.;
    \item просмотр общей длительности сериала.
\end{enumerate}

В данной работе не рассматриваются такие решения как КиноПоиск, IVI, OKKO и т. д., поскольку они являются онлайн-кинотеатрами.
Данные сервисы содержат информацию о сериалах, но их главная задача --- обеспечение возможности просмотра видео. 
Поэтому для анализа существующих решений были выбраны сервисы наиболее схожие по функционалу с разрабатываемым решением.

В таблице~\ref{tbl:cmp} представлены результаты сравнения существующих решений.

\begin{table}[H]
\begin{center}
    \caption{Таблица сравнения существующих решений}
    \label{tbl:cmp}
    \begin{tabular}{|c|c|c|c|c|c|}
        \hline
        \multirow{2}{*}{Решение} & \multicolumn{5}{c|}{Критерий}\\ \cline{2-6}
            & \makecell{Поиск} 
            & \makecell{История\\просмот-\\ров}
            & \makecell{Добавление\\в избранное}
            & \makecell{Сравнение}
            & \makecell{Общая\\длитель-\\ность} 
        \\ \hline
        myshows \cite{myshows}      & $+$ & $-$ & $+$ & $-$ & $+$\\ \hline
        shikimori \cite{shikimori}  & $+$ & $-$ & $+$ & $-$ & $-$\\ \hline
        tvguru \cite{tvguru}        & $+$ & $-$ & $+$ & $-$ & $-$\\ \hline
        kinorium \cite{kinorium}    & $+$ & $+$ & $+$ & $-$ & $+$\\ \hline
    \end{tabular}
\end{center}
\end{table}

Из таблицы~\ref{tbl:cmp} видно, что ни одно из существующих решений не удовлетворяет всем критериям.

\section{Формализация задачи}

В рамках курсовой работы необходимо разработать базу данных для хранения и обработки данных сайта, посвященного сериалам, а также реализовать веб-приложение, обеспечивающее к ней доступ.

Приложение должно работать по модели клиент-сервер. 
Серверная часть должна обеспечивать обработку запросов от клиента, а также доступ к базе данных.
Клиентская часть должна предоставлять интерфейс для отправки таких запросов.

Разрабатываемая база данных должна содержать информацию о сериалах, их сезонах и эпизодах, актерах, режиссерах, а также о зарегистрированных пользователях.

На уровне базы данных должны быть реализованы 3 роли: гость (неавторизованный пользователь), авторизованный пользователь, администратор.
На рисунке~\ref{img:usecases} представлены возможности каждой роли.

\includeimage
    {usecases} % Имя файла без расширения (файл должен быть расположен в директории inc/img/)
    {f} % Обтекание (без обтекания)
    {H} % Положение рисунка (см. figure из пакета float)
    {0.9\textwidth} % Ширина рисунка
    {Диаграмма вариантов использования} % Подпись рисунка

\section{Формализация данных}

Разрабатываемая база данных должна состоять из следующих сущностей:
\begin{itemize}[label=---]
    \item сериал;
    \item сезон;
    \item серия;
    \item актер;
    \item режиссер;
    \item пользователь;
    \item отзыв;
    \item избранное.
\end{itemize}

На рисунке~\ref{img:er} представлена ER-диаграмма сущностей проектируемой базы данных в нотации Чена.

\includeimage
    {er}
    {f}
    {H}
    {0.9\textwidth}
    {ER-диаграмма сущностей в нотации Чена}

\section{Анализ существующих баз данных}

Модель данных --- это сововкупность правил порождения структур данных в базе данных, операций над ними, а также ограничений целостности, определяющих допустимые связи и значения данных, последовательность их изменения~\cite{model-def}.

Выделяют следующие модели баз данных~\cite{db-models}:
\begin{enumerate}
    \item иерархические;
    \item сетевые;
    \item реляционные;
    \item объектно-ориентированные.
\end{enumerate}

\subsection{Иерархическая модель}

Организация данных в иерархических моделях определяется в следующих терминах~\cite{hierarchical}:
\begin{itemize}[label=---]
    \item атрибут --- наименьшая единица структуры данных;
    \item запись --- именованная совокупность атрибутов;
    \item групповое отношение --- иерархическое отношение между записями двух типов.
\end{itemize}

В иерархической модели данные организованы в виде древовидного графа с записями в виде узлов и множествами в виде ребер~\cite{db-models}.
У каждого потомка может быть только один родитель, в тоже время у родителя может быть большое количество потомков.
В связи с этим данная модель не поддерживает отношение <<многие ко многим>>. 

Уникальное значение ключа обязательно должно присутствовать в корневой записи каждого дерева. 
Ключи записей, не являющихся корневыми, должны быть уникальными только в пределах своего родителя. 
Каждая запись идентифицируется полным сцепленным ключом, который представляет собой уникальную комбинацию ключей всех записей от корня по иерархическому пути~\cite{hierarchical}.

Для каждой некорневой записи в базе данных должна быть определена родительская запись. 
При удалении родительской записи все ее дочерние записи будут автоматически удалены.

\subsection{Сетевая модель}

В сетевой модели данные организованы в виде сети, где каждый элемент может быть связан с несколькими другими элементами.
Такой подход к организации данных является расширением иерархического.
Таким образом, сетевая модель позволяет организовывать базы данных, структура которых представляется графом общего вида~\cite{model-def}.

Как и в иерархической модели обеспечивается только поддержание целостности по ссылкам~\cite{network}.

\subsection{Реляционная модель}

В реляционной модели данные организованы в виде таблиц (отношений) и все операции над базой данных сводятся к манипулированию таблицами~\cite{db-models}. 
Отношение отражает тип объекта реального мира (сущность).
Каждое отношение состоит из строк (кортежей) и столбцов (атрибутов).

Основными свойствами отношений являются~\cite{rel-properties}:
\begin{itemize}[label=---]
    \item в каждом кортеже отношения отсутствуют дубликаты, что обусловлено наличием у каждого кортежа первичного ключа. Для каждого отношения, по крайней мере, все его атрибуты вместе образуют первичный ключ;
    \item порядок атрибутов не имеет значения, поэтому для обращения к значениям атрибутов используются их имена;
    \item значения атрибутов являются атомарными, то есть не могут содержать в себе множества значений (отношения).
\end{itemize}

В реляционной модели данных существуют два основных требования целостности~\cite{rel-restricts}:
\begin{enumerate}
    \item целостность сущностей: каждый кортеж в отношении должен быть уникальным, то есть отличаться от всех остальных кортежей в этом отношении. Для этого каждое отношение должно иметь первичный ключ;
    \item целостность ссылок: каждое значение внешнего ключа, присутствующее в дочернем отношении, должно иметь соответствующее значение первичного ключа в родительском отношении.
\end{enumerate}

\subsection{Объектно-ориентированная модель}

Структура объектной модели описываются с помощью трех ключевых понятий~\cite{OOM}:
\begin{enumerate}
    \item инкапсуляция: каждый объект содержит некоторое внутреннее состояние и методы доступа к этому состоянию. Объекты представляют собой автономные сущности, отделенные от внешнего мира;
    \item наследование: возможность создания новых классов объектов на основе существующих классов, унаследовав их структуру и методы, при этом добавляя свои собственные черты;
    \item полиморфизм: различные объекты могут по-разному обрабатывать одни и те же события в зависимости от их методов.
\end{enumerate}

В объектно-ориентированной модели данные представлены в виде объектов с атрибутами и методами, что позволяет более гибко организовывать данные и использовать принципы объектно-ориентированного программирования~\cite{db-models}. 

Для обеспечения целостности объектно-ориентированный подход предлагает следующие средства~\cite{OOM}:
\begin{itemize}[label=---]
    \item автоматическое управление наследованием;
    \item возможность объявления некоторых полей данных и методов объекта доступными только самому объекту;
    \item создание процедур контроля целостности внутри объекта.
\end{itemize}

\section{Вывод}
Таким образом, была выбрана реляционная модель базы данных, поскольку реляционная модель позволяет хранить данные в виде таблиц, что обеспечивает четкую структуру и организацию данных.
Также реляционные базы данных поддерживают механизмы целостности данных, такие как ограничения целостности и транзакции, что обеспечивает защиту данных от ошибок и нежелательных изменений.
