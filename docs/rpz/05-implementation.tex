\chapter{Технологический раздел}

В данном разделе будут обоснован выбор средств реализации, представлены листинги создания таблиц и ограничений целостности базы данных, создания ролей, реализации хранимой процедуры и способа ее тестирования, а также будут приведены примеры работы программы.

\section{Выбор системы управления базами данных}

Существует большое количество реляционных СУБД, и самые популярные из них это~\cite{popular-subd} --- Oracle~\cite{oracle}, MySQL~\cite{mysql}, Microsoft SQL Server~\cite{mss}, PostgreSQL~\cite{postgres}.

Для сравнения выбранных СУБД выделены следующие критерии:
\begin{itemize}[label=---]
    \item бесплатное распространение СУБД;
    \item производительность (на основе источника~\cite{proizv});
    \item наличие опыта работы с СУБД.
\end{itemize}

Результаты сравнения выбранных СУБД по заданным критериям пред-
ставлены в таблице~\ref{tbl:subdCmp}.

\begin{table}[H]
    \begin{center}
        \caption{Сравнение выбранных СУБД}
        \label{tbl:subdCmp}
        \begin{tabular}{|c|c|c|c|}
            \hline
            \multirow{2}{*}{СУБД} & \multicolumn{3}{c|}{Критерий} \\ \cline{2-4}
                                    & \makecell{Является\\бесплатной} 
                                    & \makecell{Производительность}
                                    & \makecell{Наличие опыта\\работы с СУБД}
            \\ \hline
            Oracle &---&2&--- \\\hline
            MySQL &+&3&--- \\\hline
            Microsoft SQL Server &---&4&--- \\\hline
            PostgreSQL &+&1&+ \\\hline
        \end{tabular}
    \end{center}
\end{table}

В качестве системы управления базами данных была выбрана PostgreSQL.
Данная СУБД поддерживает все необходимые типы данных для реализации описанной выше базы данных, предоставляет возможность создания хранимых функций.
Кроме того, PostgreSQL является свободным программным обеспечением (open source), доступна бесплатно и присутствует опыт работы с данной СУБД.


\section{Средства реализации}

Для написания серверной части приложения был выбран язык программирования Golang~\cite{golang}, так как он обладает всеми необходимыми инструментами для выполнения поставленных задач. 
Golang предоставляет функционал для создания простых одностраничных серверов, а также инструменты для взаимодействия с базой данных, разработанной с использованием PostgreSQL.
В частности были использованы такие пакеты как 
<<gorilla/mux>>, реализующий маршрутизатор запросов, 
<<gorilla/sessions>> для управления сессиями, 
<<jmoiron/sqlx>> для работы с базой данных, 
<<sirupsen/logrus>>, обеспечивающий логирование, 
<<x/crypto/bcrypt>>, позволяющий хранить пароли в зашифрованном виде, 
<<net/http>>, предоставляющий клиентскую и серверную реализации HTTP, 
<<BurntSushi/toml>>, использующийся для декодирования TOML файлов.

Для реализации клиентских веб-страниц выбран язык разметки HTML~\cite{html}.

Для реализации хранимой процедуры использовалось процедурное расширение PL/pgSQL~\cite{plpgsql}.

Для автоматизации развертывания и управления приложением использовался Docker~\cite{docker}.

В качестве среды разработки использовался Visual Studio Code~\cite{vscode}.

\section{Создание таблиц}

В листингах~\ref{lst:1}~--~\ref{lst:11} представлены создание таблиц и ограничений целостности базы данных.


\lstinputlisting[firstline=1, lastline=15, caption={Создание таблицы Serials}, label={lst:1}]{../../code/database/create_tables.sql}
\lstinputlisting[firstline=16, lastline=24, caption={Создание таблицы Actors}, label={lst:2}]{../../code/database/create_tables.sql}
\lstinputlisting[firstline=25, lastline=31, caption={Создание таблицы Producers}, label={lst:3}]{../../code/database/create_tables.sql}
\lstinputlisting[firstline=32, lastline=42, caption={Создание таблицы Seasons}, label={lst:4}]{../../code/database/create_tables.sql}
\lstinputlisting[firstline=43, lastline=53, caption={Создание таблицы Episodes}, label={lst:5}]{../../code/database/create_tables.sql}
\lstinputlisting[firstline=54, lastline=66, caption={Создание таблицы Users}, label={lst:6}]{../../code/database/create_tables.sql}
\lstinputlisting[firstline=67, lastline=73, caption={Создание таблицы Favourites}, label={lst:7}]{../../code/database/create_tables.sql}
\lstinputlisting[firstline=74, lastline=83, caption={Создание таблицы Comments}, label={lst:8}]{../../code/database/create_tables.sql}
\lstinputlisting[firstline=84, lastline=93, caption={Создание таблицы Serials\_Users}, label={lst:9}]{../../code/database/create_tables.sql}
\lstinputlisting[firstline=94, lastline=102, caption={Создание таблицы Serials\_Actors}, label={lst:10}]{../../code/database/create_tables.sql}
\lstinputlisting[firstline=103, lastline=111, caption={Создание таблицы Serials\_Favourites}, label={lst:11}]{../../code/database/create_tables.sql}

\section{Создание ролей базы данных}

В листингах~\ref{lst:12}~--~\ref{lst:14} представлены создание ролей базы данных.

\lstinputlisting[firstline=1, lastline=8, caption={Создание роли guest}, label={lst:12}]{../../code/database/create_roles.sql}
\lstinputlisting[firstline=9, lastline=19, caption={Создание роли regUser}, label={lst:13}]{../../code/database/create_roles.sql}
\lstinputlisting[firstline=20, lastline=32, caption={Создание роли adminUser}, label={lst:14}]{../../code/database/create_roles.sql}

\section{Реализация хранимой функции}

В листинге~\ref{lst:15} представлена реализация хранимой функции, обеспечивающей поиск сериалов из избранного, которые можно посмотреть за выходные.

\lstinputlisting[caption={Реализация хранимой функции, обеспечивающей поиск сериалов из избранного, которые можно посмотреть за выходные}, label={lst:15}]{../../code/database/get_weekend_serials.sql}

Тестирование функции выполнялось по алгоритму, представленному на рисунке~\ref{img:testing}.

\includeimage
    {testing}
    {f}
    {H}
    {0.9\textwidth}
    {Алгоритм тестирования хранимой функции}

Код программы представлен в листинге~\ref{lst:testpy}.

\lstinputlisting[caption={Тестирование хранимой функции}, label={lst:testpy}]{../../code/database/testpy.py}


\section{Примеры работы}

На рисунках~\ref{img:demo1}~--~\ref{img:demo4} представлены примеры работы программы.

\includeimage
    {demo1}
    {f}
    {H}
    {0.9\textwidth}
    {Главная страница}

\includeimage
    {demo2}
    {f}
    {H}
    {0.9\textwidth}
    {Страница сериала: список серий}

\includeimage
    {demo3}
    {f}
    {H}
    {0.9\textwidth}
    {Страница кабинета администратора}

\includeimage
    {demo4}
    {f}
    {H}
    {0.9\textwidth}
    {Страница сравнения сериалов}

\section*{Вывод}

В данном разделе был обоснован выбор средств реализации, представлены листинги создания таблиц и ограничений целостности базы данных, создания ролей, реализации хранимой процедуры и способа ее тестирования, а также были приведены примеры работы программы.