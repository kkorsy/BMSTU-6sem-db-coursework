\chapter{Конструкторский раздел}

В данном разделе будут описаны сущности, ролевая модель и хранимая функция проектируемой базы данных.
Будет представлена диаграмма базы данных, а также описаны проектируемые ограничения целостности.

\section{Описание сущностей проектируемой базы данных}

На основании ER-диаграммы, представленной на рисунке~\ref{img:er}, проектируемая база данных должна содержать следующие таблицы:
\begin{itemize}[label=---]
    \item serials: таблица сериалов;
    \item actors: таблица актеров;
    \item producers: таблица режиссеров;
    \item seasons: таблица сезонов;
    \item episodes: таблица серий;
    \item users: таблица пользователей;
    \item favourites: таблица избранного;
    \item comments: таблица комментариев;
    \item serials\_users: таблица просмотренных пользователем сериалов;
    \item serials\_actors: таблица актеров, учавствующих в сериале;
    \item serials\_favourites: таблица сериалов в избранном.
\end{itemize}

На рисунке~\ref{img:db-diagram} представлена диаграмма проектируемой базы данных в нотации Мартина.

\includeimage
    {db-diagram}
    {f}
    {H}
    {0.9\textwidth}
    {Диаграмма базы данных}

В таблицах~\ref{tbl:serials}~--~\ref{tbl:sf} представлены имена столбцов каждой таблиц и описаны ограничения целостности.
Поля всех таблиц не могут быть пустыми.

\begin{table}[H]
    \begin{center}
        \caption{Таблица serials}
        \label{tbl:serials}
        \begin{tabular}{|l|c|c|}
            \hline
            Имя столбца & Описание & Ограничение \\ \hline
            s\_id & \makecell{уникальный иденти-\\фикатор сериала} & является уникальным \\ \hline
            s\_idProducer & \makecell{идентификатор\\режиссера сериала} & --- \\ \hline
            s\_name & название сериала & --- \\ \hline
            s\_description & описание сериала & --- \\ \hline
            s\_year & год выхода сериала & \makecell{целое число,\\большее 1800} \\ \hline
            s\_genre & жанр сериала & --- \\ \hline
            s\_rating & рейтинг сериала & вещественное число от 0 до 10 \\ \hline
            s\_seasons & \makecell{количество сезонов\\в сериале} & целое число, большее 0 \\ \hline
            s\_state & статус сериала & \makecell{принимает 2 возможных значения:\\<<завершен>>, <<продолжается>>}\\ \hline
        \end{tabular}
    \end{center}
\end{table}

\begin{table}[H]
    \begin{center}
        \caption{Таблица actors}
        \label{tbl:actors}
        \begin{tabular}{|l|c|c|}
            \hline
            Имя столбца & Описание & Ограничение \\ \hline
            a\_id & \makecell{уникальный иденти-\\фикатор актера} & является уникальным \\ \hline
            a\_name & имя актера & --- \\ \hline
            a\_surname & фамилия актера & --- \\ \hline
            a\_gender & пол актера & \makecell{принимает 2 возможных значения:\\<<мужской>>, <<женский>>} \\ \hline
            a\_bdate & дата рождения актера & --- \\ \hline
        \end{tabular}
    \end{center}
\end{table}

\begin{table}[H]
    \begin{center}
        \caption{Таблица producers}
        \label{tbl:producers}
        \begin{tabular}{|l|c|c|}
            \hline
            Имя столбца & Описание & Ограничение \\ \hline
            p\_id & \makecell{уникальный иденти-\\фикатор режиссера} & является уникальным \\ \hline
            p\_name & имя режиссера & --- \\ \hline
            p\_surname & фамилия режиссера & --- \\ \hline
        \end{tabular}
    \end{center}
\end{table}

\begin{table}[H]
    \begin{center}
        \caption{Таблица seasons}
        \label{tbl:seasons}
        \begin{tabular}{|l|c|c|}
            \hline
            Имя столбца & Описание & Ограничение \\ \hline
            ss\_id & \makecell{уникальный иденти-\\фикатор сезона} & является уникальным \\ \hline
            ss\_idSerial & идентификатор сериала & ---\\ \hline
            ss\_name & название сезона & ---\\ \hline
            ss\_num & номер сезона & целое число, большее 0\\ \hline
            ss\_cntEpisodes & \makecell{количество эпизодов\\в сезоне} & целое число, не меньшее 0\\ \hline
            ss\_date & дата выхода сезона & ---\\ \hline
        \end{tabular}
    \end{center}
\end{table}

\begin{table}[H]
    \begin{center}
        \caption{Таблица episodes}
        \label{tbl:episodes}
        \begin{tabular}{|l|c|c|}
            \hline
            Имя столбца & Описание & Ограничение \\ \hline
            e\_id & уникальный идентификатор серии & является уникальным \\ \hline
            e\_idSeason & идентификатор сезона & ---\\ \hline
            e\_name & название серии & ---\\ \hline
            e\_num & номер серии & целое число, большее 0\\ \hline
            e\_duration & продолжительность серии & ---\\ \hline
            e\_date & дата выхода серии & ---\\ \hline
        \end{tabular}
    \end{center}
\end{table}

\begin{table}[H]
    \begin{center}
        \caption{Таблица users}
        \label{tbl:users}
        \begin{tabular}{|l|c|c|}
            \hline
            Имя столбца & Описание & Ограничение \\ \hline
            u\_id & \makecell{уникальный иденти-\\фикатор пользователя} & является уникальным\\ \hline
            u\_idFavourites & \makecell{идентификатор\\избранного} & ---\\ \hline
            u\_login & логин пользователя & ---\\ \hline
            u\_password & \makecell{хеш пароля\\пользователя} & ---\\ \hline
            u\_role & роль пользователя & \makecell{принимает 2 возможных\\значения: <<user>>, <<admin>>}\\ \hline
            u\_name & имя пользователя & ---\\ \hline
            u\_surname & фамилия пользователя & ---\\ \hline
            u\_gender & пол пользователя & \makecell{принимает 2 возможных\\значения: <<мужской>>,\\<<женский>>}\\ \hline
            u\_date & \makecell{дата рождения\\пользователя} & ---\\ \hline
        \end{tabular}
    \end{center}
\end{table}

\begin{table}[H]
    \begin{center}
        \caption{Таблица favourites}
        \label{tbl:favourites}
        \begin{tabular}{|l|c|c|}
            \hline
            Имя столбца & Описание & Ограничение \\ \hline
            f\_id & \makecell{уникальный иденти-\\фикатор избранного} & является уникальным\\ \hline
            f\_cntSerials & \makecell{количество сериалов\\в избранном} & целое число, не меньшее 0\\ \hline
        \end{tabular}
    \end{center}
\end{table}

\begin{table}[H]
    \begin{center}
        \caption{Таблица comments}
        \label{tbl:comments}
        \begin{tabular}{|l|c|c|}
            \hline
            Имя столбца & Описание & Ограничение \\ \hline
            c\_id & \makecell{уникальный иденти-\\фикатор комментария} & является уникальным\\ \hline
            c\_idUser & идентификатор пользователя & ---\\ \hline
            c\_text & текст комментария & ---\\ \hline
            c\_date & дата оставления комментария & ---\\ \hline
        \end{tabular}
    \end{center}
\end{table}

\begin{table}[H]
    \begin{center}
        \caption{Таблица serials\_users}
        \label{tbl:su}
        \begin{tabular}{|l|c|c|}
            \hline
            Имя столбца & Описание & Ограничение \\ \hline
            su\_id & \makecell{уникальный иденти-\\фикатор сериалов, просмотренных\\пользователем} & является уникальным\\ \hline
            su\_idSerial & идентификатор сериала & ---\\ \hline
            su\_idUser & идентификатор пользователя & ---\\ \hline
            su\_lastSeen & дата последнего просмотра & ---\\ \hline
        \end{tabular}
    \end{center}
\end{table}

\begin{table}[H]
    \begin{center}
        \caption{Таблица serials\_actors}
        \label{tbl:sa}
        \begin{tabular}{|l|c|c|}
            \hline
            Имя столбца & Описание & Ограничение \\ \hline
            sa\_id & \makecell{уникальный иденти-\\фикатор актеров, учавствующих\\в сериале} & является уникальным\\ \hline
            sa\_idSerial & идентификатор сериала & ---\\ \hline
            sa\_idActor & идентификатор актера & ---\\ \hline
        \end{tabular}
    \end{center}
\end{table}

\begin{table}[H]
    \begin{center}
        \caption{Таблица serials\_favourites}
        \label{tbl:sf}
        \begin{tabular}{|l|c|c|}
            \hline
            Имя столбца & Описание & Ограничение \\ \hline
            sf\_id & \makecell{уникальный иденти-\\фикатор сериалов в избранном} & является уникальным\\ \hline
            sf\_idSerial & идентификатор сериала & ---\\ \hline
            sf\_idFavourites & идентификатор избранного & ---\\ \hline
        \end{tabular}
    \end{center}
\end{table}


\section{Описание проектируемой ролевой модели}

В предыдущем разделе были выделены 3 роли: гость (неавторизованный пользователь), авторизованный пользователь, администратор.
В таблице~\ref{tbl:roles} описаны возможности каждой роли. Введены следующие обозначения: C --- CREATE, R --- READ, U --- UPDATE, D --- DELETE.

\begin{table}[H]
    \begin{center}
        \caption{Возможности ролей}
        \label{tbl:roles}
        \begin{tabular}{|c|c|c|c|}
            \hline
            \multirow{2}{*}{Таблица} & \multicolumn{3}{c|}{Роль} \\ \cline{2-4}
                                    & \makecell{Гость} 
                                    & \makecell{Авторизованный\\пользователь}
                                    & \makecell{Администратор}
            \\ \hline
            serials & R & R & C, R, U, D\\ \hline
            actors & R & R & C, R, U, D\\ \hline
            producers & R & R & C, R, U, D\\ \hline
            seasons & R & R & C, R, U, D\\ \hline
            episodes & R & R & C, R, U, D\\ \hline
            users & C & R, U, D & R, U, D\\ \hline
            favourites & - & R, U, D & R, U, D\\ \hline
            comments & - & C, R, U, D & R\\ \hline
            serials\_users & - & C, R, D & -\\ \hline
            serials\_actors & - & - & C, R, U, D\\ \hline
            serials\_favourites & - & C, R, D & -\\ \hline
        \end{tabular}
    \end{center}
\end{table}

\section{Описание проектируемой хранимой функции}

В аналитическом разделе одним действий пользователя была возможность посмотреть сериалы из избранного, которые можно посмотреть за выходные.
Для реализации данного функционала было принято решение спроектировать функцию.

На рисунке~\ref{img:funcScheme} представлена схема алгоритма поиска сериалов из избранного, которые можно посмотреть за выходные.

\includeimage
    {funcScheme}
    {f}
    {H}
    {0.9\textwidth}
    {Схема алгоритма поиска сериалов из избранного, которые можно посмотреть за выходные}

\section{Вывод}

В данном разделе были описаны сущности, ролевая модель и хранимая функция проектируемой базы данных.
Была представлена диаграмма базы данных, а также описаны проектируемые ограничения целостности.
